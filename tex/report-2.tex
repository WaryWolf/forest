\documentclass[11pt,a4paper]{report}
\usepackage{a4wide}
\usepackage{color}
\usepackage{graphicx}
% Default margins are too wide all the way around. I reset them here
%\setlength{\topmargin}{-.5in}
%\setlength{\textheight}{9in}
%\setlength{\oddsidemargin}{.125in}
%\setlength{\textwidth}{175mm}
\setlength{\parindent}{0cm}
\begin{document}
\title{COSC3500\\Serial Implementation Report}
\author{Anthony Vaccaro\\University of Queensland}

%\today
\maketitle
\begin{abstract}
This report introduces the concept of modelling forest fires with cellular
automata, and discusses an implementation of said model.
\end{abstract}

\newpage
\tableofcontents
\newpage

\chapter{1: Initial Research and Serial Implementation}

\newpage
\section{The Forest-Fire Model}

The Forest-Fire model is a cellular automaton-based model of a wild fire
spreading through a forest of trees or other plants. The model has a few
variants but typically follows these rules:

\begin{itemize}
\item An empty cell becomes a tree with probability p.
\item A tree becomes a burning tree with probability f.
\item A tree becomes a burning tree if its neighbours are burning.
\item A burning tree becomes an empty cell.
\end{itemize}
Where variables p and f can be adjusted to produce different patterns.

These simple rules can produce complex patterns and have lead to some research
into the area of modelling forest fires.

\newpage
\section{Algorithm used to represent model}

The algorithm chosen for this report and corresponding program is based on the
rules given above, with a few minor adjustments:

\begin{itemize}
\item Burning trees take l timesteps to become empty.
\item Trees with burning neighbours have probability c to catch fire.
\end{itemize}

This gives provides the simulation with four main variables to adjust - p, f,
l, and c.

\includegraphics{tex/flowchart}

\newpage
\section{Implementation details}

The implementation was chosen to be written in C, based on the student's past
experience with programming. The implementation did not require any complex
programming techniques, although a few key points are worth discussing.

As much of this simulation relies on probabilities, the use of a random number
generator requires some thought. the \texttt{random()} function of the C
standard library was selected due to its ease of use and availability, but as
it is only a pseudo-random number generator, perhaps this could be changed in
later efforts. There are websites that provide "true randomness" based on
atmospheric noise\cite{random}, and perhaps data from this service could be
used and compared against the output of \texttt{random()}.

Exporting the output of the program was done via the use of
libpng\cite{libpng}, which can easily convert 2d arrays into pictures. Sample
code was used from a third party website in order to assist with this
process\cite{png}. After each timestep was completed, the grid of cells was
exported to an image, with pixels being coloured based on their status.

\newpage
\section{Verification of program's output correctness}

The implementation currently uses \texttt{random()} for its random number
generation, and this can be made deterministic by using a predefined seed with
the \texttt{srandom()} function. This will be used to compare the results of
later optimisations and multiprocessing techniques.

\newpage
\section{Algorithm's performance while scaling with forest size}

As the algorithm operates on neighbours, which remain constant with an
increasing grid size, performance should scale linearly with number of cells in
the grid.

\begin{verbatim}

$ time ./forest 100 0 100
real    0m1.179s
user    0m0.433s
sys     0m0.046s

$ time ./forest 200 0 100
real    0m2.524s
user    0m1.802s
sys     0m0.113s

$ time ./forest 300 0 100

real    0m5.395s
user    0m4.500s
sys     0m0.172s



\end{verbatim}

\newpage
\section{Further Considerations}

\newpage

\chapter{2: Optimisation and Parallel Implementation}

\newpage

\section{Optimisation of Serial Algorithm}

Before commencing with parallelisation of the \texttt{forest} program, the code
was examined for optimisations that could be made. An initial profile of the
program\footnote{The program was compiled without any optimisation flags, and
run on a 400 by 400 cell grid for 1000 timesteps.} with the \texttt{gprof} tool gave the following output:

\begin{verbatim}
Flat profile:

Each sample counts as 0.01 seconds.
  %   cumulative   self              self     total
 time   seconds   seconds    calls  ms/call  ms/call  name
 51.29      6.30     6.30     1000     6.30     8.04  out_png
 28.22      9.77     3.47 160000000     0.00     0.00  cell_auto
  8.32     10.79     1.02     1000     1.02     1.38  save_png_to_file
  6.28     11.57     0.77     1000     0.77     4.24  timestep
  5.83     12.28     0.72 320000000     0.00     0.00  pixel_at
  0.29     12.32     0.04                             print_usage
  0.00     12.32     0.00        1     0.00     0.00  alloc_forest
  0.00     12.32     0.00        1     0.00     0.00  parse_args

\end{verbatim}

This profile showed that approximately 65\% of execution time was spent
creating the PNG images and writing them to disk.

From this profile, two immediate conclusions were drawn: There should be an
alternate output method that does not incur any I/O overhead, and the PNG
output module should be optimimised.

a "null" output module was implemented, which allowed for benchmarking of the
code without worrying about the performance of third party libraries, or I/O.
Terminal output via the \texttt{ncurses} library was also implemented, for easy
examination of code output changes.

The \texttt{timestep()} function was later moved into \texttt{main()} to save
unnecessary function call overhead.

\subsection{PNG Module}

Inside the PNG module, several immediate optimisations were recognised: the
simple function \texttt{pixel\_at()} could be easily inlined, and the
intermediary struct \texttt{bitmap} was unnecessary and could be bypassed
completely. The module was rewritten and showed a considerable
performance increase:


\begin{verbatim}
Flat profile:

Each sample counts as 0.01 seconds.
  %   cumulative   self              self     total
 time   seconds   seconds    calls  ms/call  ms/call  name
 51.53      5.11     5.11     1000     5.11     5.11  out_png
 40.72      9.15     4.04 160000000     0.00     0.00  cell_auto
  7.88      9.93     0.78     1000     0.78     4.82  timestep
  0.10      9.94     0.01                             main
  0.00      9.94     0.00        1     0.00     0.00  alloc_forest
  0.00      9.94     0.00        1     0.00     0.00  parse_args

\end{verbatim}

As can be seen from the above profile, the rewritten \texttt{out\_png()}
function takes approximately 5ms per call, as opposed to 8ms in the original
version. The overhead from unnecessary function calls was removed by combining
procedures into one function, and the total execution time spent on output was
now 51\%, down from 65\%.

\subsection{Memory Allocation}

Initially, the two-dimensional grid that was used to store cell data was
allocated using three tiers of pointers: a top-level array containing double
pointers, inner arrays of pointers, and finally structs were allocated
individually.

This presented two problems: examining a cell would require four or even five
memory lookups to dereference all the pointers, and with individually allocated
cell structs, the physical location of neighbouring cells in memory could be
quite distant, resulting in cache misses, particularly with larger grids.

By switching to a one-dimensional array that was allocated in one call to
\texttt{malloc()}, both of these problems were solved. Performance was shown to 
increase, although this was not measured as other optimisations took place
around this time.

\section{Parallelisation}

After the code was found to be sufficiently optimised in a serial environment,
methods to improve performance via parallelisation were investigated.

The main iteration loop over cells in the grid was reimplemented with two grids
instead of one, so that changes could be made to a "new" grid while reading an
"old" grid. This solved a problem with correctness in early versions of the
program, where fires would spread in one direction more than others, due to
examining neighbouring cells that were from the current timestep rather than
the previous timestep.

After moving to the two-grid implementation, iterating over one section of the
grid did not depend on other sections, and the loop could easily be
parallelised with OpenMP. Using MPI to further parallelise via domain
decomposition was also identified as a method to increase performance.

\subsection{OpenMP}

OpenMP allows for simple division of problems into smaller chunks which can be
processed independently. This is performed via multithreading.

OpenMP is implemented via the use of \texttt{pragma} directives added before
sections of code. To parallelise the main iteration loop in the \texttt{forest}
program, the \texttt{pragma omp parallel for} directive was added before the
loop initialisation, as follows:

\begin{verbatim}
    for (i = 0; i < f->simLength; i++) {
            #pragma omp parallel for
            for (x = 0; x < s; x++) {
                for (int y = 0; y < t; y++) {
                    int rand;
                    random_r(&stuff[omp_get_thread_num()], &rand);
                    cell_auto(f, x, y, f->oldGrid[(y * t) + x].status, &rand);
                    }
                }
\end{verbatim}

This split the loop at each row, giving each thread a row of the grid to
operate on.

Initial performance when using OpenMP showed a significant performance
decrease. Using the \texttt{OMP\_NUM\_THREADS} environment variable to limit
exectution to four threads showed more than a tenfold increase in execution time compared to
execution when using one thread:

\begin{verbatim}
$ export OMP_NUM_THREADS=1
$ time ./forest 400 400 null 1000
6.207 sec

$ export OMP_NUM_THREADS=4
$ time ./forest 400 400 null 1000
84.831 sec
\end{verbatim}

While examining the code, it was found that the \texttt{random()} function
used was not multithread-capable, as it relies on a global buffer to hold state
information between calls. However, the glibc library provides a reentrant random number
generator which is capable of being used in a multithreaded environment, called
\texttt{random\_r()}. The code was modified to use \texttt{random\_r()} with
individual state buffers for each thread, and performance was shown to scale
with threads:

\begin{verbatim}
$ export OMP_NUM_THREADS=1
$ time ./forest 400 400 null 1000
5.571 sec

$ export OMP_NUM_THREADS=2
$ time ./forest 400 400 null 1000
4.613 sec

$ export OMP_NUM_THREADS=4
$ time ./forest 400 400 null 1000
3.206 sec
\end{verbatim}

These tests were performed on a desktop computer with an Intel i7 2600k CPU.
When switching to a node on the Barrine cluster, performance still degraded
when using multiple threads:

\begin{verbatim}
forest with 400 x 400 grid on 1 threads took: 6.627 sec
forest with 400 x 400 grid on 2 threads took: 12.016 sec
forest with 400 x 400 grid on 4 threads took: 8.013 sec
forest with 400 x 400 grid on 8 threads took: 7.210 sec
\end{verbatim}

The cause of this performance decrease is as of yet unexplained, however it
could be due to differing CPU architectures. The Barrine node used has an Intel
Xeon L5520 CPU, which is from an earlier generation than the desktop CPU used.


\subsection{MPI}

Using MPI to parallelise programs involves running multiple instances of the
same program on multiple computers, or nodes, and communicating between each
node to transfer data. Each node running the MPI-enabled program is known as a
"rank".

A suitable method for improving performance with MPI would be to divide the
grid into multiple smaller sections, and have each rank process one section. At
the end of each timestep, each rank could pass their section to neighbouring
ranks, allowing an updated copy of the entire grid to exist on each rank. Then,
processing could continue. OpenMP could be used to divide the work on each rank
up into smaller subsections, allowing for all CPUs on a rank to be utilised
fully.

Unfortunately an MPI implementation was not written due to time constraints,
however, it is believed that this method of domain decomposition would have
improved performance when using multiple ranks.


\newpage
\section{Verification of implementation output}

To ensure that the program's output could be verified, it had to be able to
produce deterministic output. A verification output method was added, which
altered the seeding method for random number generation to use a static seed.
The module examines the contents of the cell grid at the final timestep of the
simulation, and produces an MD5 checksum of the contents. This was implemented
using glibc's \texttt{crypt()} function.

If the program's output can be determined at runtime, it should be possible to
reproduce the same output by running the program with the same arguments,
multiple times. This was found to be the case when using one OpenMP thread, but
when using multiple threads, the program's output changed between runs:


\begin{verbatim}

$ export OMP_NUM_THREADS=1
$ ./forest 240 240 verify 500
Grid checksum is: $1$cosc3500$Zz63shoT031bnGE.f1UVF/
$ ./forest 240 240 verify 500
Grid checksum is: $1$cosc3500$Zz63shoT031bnGE.f1UVF/
$ ./forest 240 240 verify 500
Grid checksum is: $1$cosc3500$Zz63shoT031bnGE.f1UVF/

$ export OMP_NUM_THREADS=2
$ ./forest 240 240 verify 500
Grid checksum is: $1$cosc3500$q.T.ss7LAG/oZpycVU9Ig0
$ ./forest 240 240 verify 500
Grid checksum is: $1$cosc3500$ibx89tXbp/9Jer53wZkPH.
$ ./forest 240 240 verify 500
Grid checksum is: $1$cosc3500$7R05CaitedeiVLqruuDFw0

\end{verbatim}

This issue is as of yet unexplained.


\section{Test Plan}

The \texttt{forest} program's output should be made deterministic when working
with multiple threads, but once this has been accomplished, it can be run on
large grids for long simulation lengths.

One area which would be worthwhile exploring is the number of trees and burning
trees in a forest over time. As fires burn more rapidly when there are more
trees, it would be interesting to see the effect of fires that occur in
sparsely-populated forests, compared to densely populated forests. Collection
of these statistics is easily accomplished with the current program, and they
could be graphed using third-party software.


\begin{thebibliography}{99}

\bibitem{png}
"Write a PNG file using C and libpng", 2010 Ben Bullock
http://www.lemoda.net/c/write-png/

\bibitem{libpng}
"libpng Home Page", 2013 Greg Roelofs
http://www.libpng.org/pub/png/libpng.html

\bibitem{random}
"RANDOM.ORG - True Random Number Service", 2012 RANDOM.ORG
http://www.random.org/

\end{thebibliography}
\end{document}
    
    
